% sudo apt install texlive-lang-italian
% sudo apt install texlive-fonts-extra
% sudo apt-get install texlive-bibtex-extra biber
\documentclass[a4paper,11pt]{article}

\usepackage[utf8]{inputenc}
\usepackage[T1]{fontenc}
\usepackage[italian]{babel}
\usepackage{graphicx}

%\usepackage[backend=biber]{biblatex}
\usepackage{cv} % backbone style
\usepackage{booktabs}
%\usepackage{fontawesome} % beautiful icons!!
\usepackage{fontawesome5} % very beautiful icons!!
\usepackage{pifont} % more styles for bullet
\usepackage[dvipsnames]{xcolor}
\usepackage[colorlinks=true, allcolors=Blue]{hyperref}

\usepackage{siunitx}
\usepackage{tikz}
\newcommand*{\priority}[1]{\begin{tikzpicture}[scale=0.15]%
    \draw[color=Blue] (0,0) circle (1);
    \fill[fill opacity=0.5,fill=Blue] (0,0) -- (90:1) arc (90:90+#1*3.6:1) -- cycle;
    \end{tikzpicture}}


% Change color to blue
\def\headcolor{\color[rgb]{0,0,0.5}}
% Space before section headings
\titlespacing{\section}{0pt}{2ex}{1ex}

\graphicspath{{./img/}}

\ifdefined\signed
  \newcommand{\SignatureImage}[2][]{%
    \IfFileExists{#2}{%
      \includegraphics[#1]{#2}%
    }{%
      \hfill\makebox[2.0in]{\hrulefill}
    }%
  }%
\else
  \newcommand{\SignatureImage}[2][]{\hfill\makebox[2.0in]{\hrulefill}}
\fi

\name{Nico Curti}
\image{io.png}
\info{
 \faUniversity      & Dept. of Experimental, Diagnostic and Specialty Medicine of Bologna University\\
 \faMapMarker*      & Bologna (Italy)\\
 \faPhone           & +39 333 997 93 99\\
 \faPaperPlane      & nico.curti2@unibo.it\\
 \faGithub          & \url{https://github.com/Nico-Curti}\\
 \faOrcid           & \url{https://orcid.org/0000-0001-5802-1195}\\
}

\pagestyle{fancy}
\lhead{Curriculum vitae e scientifico-professionale}
\rhead{Nico Curti}
\rfoot{\thepage}
\cfoot{}
\renewcommand{\headrulewidth}{0.4pt}
\newcommand{\quotes}[1]{``#1''}
\newcommand{\itemicon}[2]{\item[{\includegraphics[scale=#1]{#2}}]}
\newcommand{\icon}[2]{\includegraphics[scale=#1]{#2}}


\begin{document}

\maketitle

\section*{\scshape{Storia accademica e scientifica}}

\justifying

Nico Curti nasce a Cattolica (RN, Italia) il 28/09/1992.
Dopo aver conseguito la laurea triennale in Fisica nel 2014 presso L'Università di Bologna (BO, Italia), consegue la laurea specialistica in \emph{Fisica Applicata} (a beni culturali, ambientali, biologia e medicina) con il voto di 110/110 e lode, presentando il lavoro di tesi dal titolo \emph{Implementazione e benchmarking dell'algoritmo QDANet PRO per l'analisi di Big Data genomici}, svolto sotto la supervisione del Prof. Daniel Remondini e il Prof. Gastone Castellani dell'Università di Bologna.
Durante il lavoro di tesi, il Dr. Curti ha avuto modo di approfondire le proprie competenze informatiche, focalizzando il suo lavoro sull'implementazione ed ottimizzazione di algoritmi per applicazioni su \emph{high-performance-computers}.

Successivamente, nel 2019, consegue il dottorato di ricerca in Fisica, difendendo la tesi dal titolo \quotes{Implementation and optimization of algorithms in Biological Big Data Analytics}, sotto la supervisione del Prof. Daniel Remondini, Prof. Gastone Castellani e Prof. Armando Bazzani.
Durante il lavoro di dottorato il Dr. Curti ha approfondito le sue competenze nel campo dell'\emph{intelligenza artificiale} e \emph{machine learning}, studiando i moderni modelli a rete neurale \emph{deep learning} e le più efficienti tecniche algoritmiche per l'analisi dati.
In particolare, il Dr. Curti si è occupato dell'analisi di Big Data genomici mediante nuove tecniche di \emph{machine learning} e dell'applicazione di modelli a super risoluzione, object detection e segmentazione su immagini di natura bio-medicale.
Durante il suo dottorato il Dr. Curti ha avuto l'occasione di collaborare attivamente con numerosi gruppi di ricerca di altri atenei ed è intervenuto nell'elaborazione dati di numerosi progetti con enti privati.

Dal 2019 fino ad oggi è stato assegnista di ricerca presso il Dipartimento di Medicina Specialistica, Diagnostica e Sperimentale dell'Università di Bologna, sotto la supervisione del Prof. Enrico Giampieri, della Prof.ssa Emanuela Marcelli e del Prof. Gastone Castellani.
Durante questo periodo ha avuto modo di collaborare attivamente in numerosi progetti di carattere medico, affiancato da diversi membri delle unità ospedaliere.
In particolare, il Dr. Curti ha lavorato all'analisi di immagini istopatologiche (WSI) per la segmentazione e caratterizzazione di tessuti e cellule, sviluppando \emph{Decision Support System}s in grado di affiancare la metodologia di valutazione tradizione durante la pratica clinica.
Inoltre ha collaborato con l'unità ospedaliera di Oftalmologia allo sviluppo di una metodologia completamente automatizzata per la valutazione di immagini acquisite mediante lampada a fessura.
Il Dr. Curti ha svolto numerose ricerche ed applicazioni nel campo dell'\emph{intelligenza artificiale} applicate a immagini radiologiche, comprensive di immagini CT, PET e MRI, realizzando modelli per la segmentazione automatica delle aree tumorali e successiva caratterizzazione mediante elaborazione di features radiomiche.


\newpage


\section*{\scshape{Titoli di Studio}}

\hspace*{-0.5cm}
\begin{tabular}{lp{4cm}lp{6cm}}
  \icon{0.05}{diploma.png}       \quad 2011 & Diploma                                         & Liceo Scientifico A. Volta & \\
  \icon{0.05}{graduationcap.png} \quad 2014 & Laurea Triennale in\newline Fisica              & Università di Bologna      & \emph{Integrazione di misure NMR e microscopiche per la descrizione quantitativa degli effetti di stress esterni su colture cellulari} \\
  \icon{0.05}{degree.png}        \quad 2016 & Laurea Specialistica in Fisica Applicata        & Università di Bologna      & \emph{Implementazione e benchmarking dell'algoritmo QDANet PRO per l'analisi di Big Data genomici} \\
  \icon{0.05}{phd.png}           \quad 2019 & Dottorato in Fisica\newline(Fisica Applicata)   & Università di Bologna      & \emph{Implementation and optimization of algorithms in Biological Big Data Analytics} \\
\end{tabular}


\vspace*{0.5cm}
\section*{\scshape{Premi, Assegni di Ricerca e riconoscimenti accademici}}

\hspace*{-0.5cm}
\begin{tabular}{llp{12cm}}

  \icon{0.05}{grant.png} \quad 2016 & Assegno di ricerca & \emph{Big Data Analytics di dati genomici e sociali high-throughput in ambiente HPC} \\
  \icon{0.05}{badge.png} \quad 2017 & Premio nazionale   & \emph{Premio Nazionale Giulia Vita Finzi} per la miglior tesi di laurea magistrale su attività di ricerca e sviluppo nell'ambito del calcolo dell'INFN \\
  \icon{0.05}{grant.png} \quad 2018 & Assegno di ricerca & \emph{Applicazione di algoritmi di machine learning nel contesto della comunicazione medico-paziente, all'interno del progetto FILOBLU} (INFN) \\
  \icon{0.05}{grant.png} \quad 2019 & Assegno di ricerca & \emph{Integrazione di dati clinici e multi-omici per la cura dei pazienti con patologie complesse e multisettoriali} (DIMES) \\
  \icon{0.05}{grant.png} \quad 2020 & Assegno di ricerca & \emph{Computer Vision ed Intelligenza artificiale per l'armonizzazione e l'analisi di imaging medico e dati multiomici} (DIMES) \\
  \icon{0.05}{grant.png} \quad 2021 & Assegno di ricerca & \emph{Machine learning ed Intelligenza artificiale per l'armonizzazione e l'analisi di dati multiomici del progetto HARMONY-PLUS} (DIMES) \\
  \icon{0.05}{badge.png} \quad 2022 & Start Cup          & \emph{Finalista} della Start Cup - Emilia-Romagna 2022 con il progetto imprenditoriale \emph{HIDRA - Human Interactions in Disease Records and Atlases}\\

\end{tabular}


\vspace*{0.5cm}
\section*{\scshape{Didattica}}

\hspace*{-0.5cm}
\begin{tabular}{lp{14cm}}

  \icon{0.05}{education.png} \quad 2013                & Volontario presso la Secondary School del villaggio di Guandumehhy del distretto di Mbulu (Tanzania) come docente di matematica e fisica. \\

  \icon{0.05}{education.png} \quad 2016                & Tutor presso la Scuola di Scienze dell'Università di Bologna per l'attività formativa dal titolo \emph{Analisi dati della fisica} per il corso di studi in fisica (16 ore). \\

  \icon{0.05}{education.png} \quad 2020\textemdash2021 & Assistente del corso di \emph{Physical Methods of Biology} - 87995 (6 cfu) del prof. Castellani per il corso di Laurea Magistrale in Physics dell'Università di Bologna. \\

  \icon{0.05}{education.png} \quad 2020\textemdash2022 & Professore a contratto presso la Scuola di Scienze dell'Università di Bologna per il corso di laurea in Infermieristica (abilitante alla professione sanitaria di infermiere) per il corso di \emph{Fisica Applicata} - 58200 (24 ore). \\

  \icon{0.05}{education.png} \quad 2022                & Corso elettivo \emph{AI4Medicine} presso il Dept. of Experimental, Diagnostic and Specialty Medicine of Bologna University. \\

\end{tabular}


\vspace*{0.5cm}
\section*{\scshape{Partecipazione a conferenze}}

\begin{itemize}

  \itemicon{0.03}{conference.png} Partecipazione alla conferenza \emph{Problems in discrete dynamics: from biochemical systems to rare events, networks, clustering and related topics - II Edition}, tenutasi dal 05/10/2017 al 07/10/2017, nella quale ho presentato il lavoro dal titolo \emph{Learning by message-passing in networks of discrete synapses the traffic congestion prediction}.

  \itemicon{0.03}{conference.png} Partecipazione alla conferenza \emph{EuroPar 2018}, tenutasi dal 27/08/2018 al 31/08/2018, nella quale ho presentato il lavoro dal titolo \emph{Cross-Environment comparison of a bioinformatics pipeline: perspectives for hybrid computations}.

  \itemicon{0.03}{conference.png} Partecipazione alla conferenza \emph{INFN BioPhys and PlexNet}, tenutasi dal 10/09/2018 al 12/09/2018, nella quale ho presentato il lavoro dal titolo \emph{Cross-Environment comparison of a bioinformatics pipeline: perspectives for hybrid computations}.

  \itemicon{0.03}{conference.png} Partecipazione alla conferenza \emph{INFN BioPhys and PlexNet}, tenutasi dal 24/09/2019 al 26/09/2019, nella quale ho presentato il lavoro dal titolo \emph{Introducing the Complex Human Interactions in MEdical Records and Atlases Network - CHIMERA}.

  \itemicon{0.03}{conference.png} Partecipazione alla conferenza \emph{AIM - Live Meeting}, tenutasi il 16/10/2020, nella quale ho presentato il lavoro dal titolo \emph{Automatic Pipeline for Identification of Ground Glass Opacities in CT Images of COVID-19 Affected Patients}.

  \itemicon{0.03}{conference.png} Partecipazione alla \emph{Notte dei Ricercatori} tenutasi il 27/11/2020, nella quale ho presentato il lavoro dal titolo \emph{Intelligenza artificiale in medicina - AI vs COVID-19}.

  \itemicon{0.03}{conference.png} Partecipazione alla conferenza \emph{Next-AIM - kick-off meeting}, tenutasi il 18/02/2022, nella quale ho presentato il lavoro dal titolo \emph{Super Resolution in Medical Images}.

  \itemicon{0.03}{conference.png} Partecipazione alla conferenza \emph{Complex System}, tenutasi il 24/03/2022, nella quale ho presentato il lavoro dal titolo \emph{Effectiveness of biological inspired neural network models in learning and patterns memorization}.

  \itemicon{0.03}{conference.png} Partecipazione al seminario \emph{Medico per un giorno}, tenutasi il 01/04/2022, nella quale ho presentato il lavoro dal titolo \emph{Medicina del Futuro: Intelligenza Artificiale e Immagini Mediche}.

  \itemicon{0.03}{conference.png} \textbf{Invited Speaker} al seminario \emph{Intelligenza Artificiale in Senologia Parte II: Principi e Potenzialità}, tenutasi il 16/02/2022, nella quale ho presentato il lavoro dal titolo \emph{Intelligenza Artificiale: i diversi approcci}.

  \itemicon{0.03}{conference.png} \textbf{Invited Speaker} alla conferenza \emph{Radiomics toolbox: Workflow and quality management}, tenutasi dal 07/09/2022 al 09/09/2022, nella quale ho presentato il lavoro dal titolo \emph{Biophysics inspired neural network}.

\end{itemize}


\vspace*{0.5cm}
\section*{\scshape{Corsi di formazione}}

\begin{itemize}

  \itemicon{0.04}{code.png} Partecipazione al corso di formazione presso l'INFN International School di Bertinoro dal titolo \emph{Eighth I.N.F.N. International School on architectures, tools and methodologies for developing efficient large scale scientific computing applications}, tenutosi dal 24/10/2016 al 29/10/2016.

  \itemicon{0.04}{code.png} Partecipazione a \emph{Intel-Code Modernization Workshop Rome}, tenutosi dal 23/05/2017 al 24/05/2017.

  \itemicon{0.03}{conference.png} Partecipazione alla conferenza \emph{Due seminari sul tema dell'Intelligenza Artificiale}, tenutasi il 13/06/2018, presentata dal Prof. Gianni Zanarini e dalla Prof.ssa Paola Mello.

  \itemicon{0.04}{code.png} Partecipazione al corso di formazione presso il centro INFN CNAF di Bologna dal titolo \emph{Implementazione di Sistemi di Gestione per la Sicurezza delle Informazioni in conformità alla Norma UNI CEI ISO/IEC 27001}, tenutosi dal 15/07/2019 al 18/07/2019.

  \itemicon{0.03}{conference.png} Partecipazione alla \emph{Call for Business Plan 2021} dell'Università di Bologna.

  \itemicon{0.03}{conference.png} Partecipazione alla \emph{StartCup 2022} della regione Emilia-Romagna.

\end{itemize}


\vspace*{0.5cm}
\section*{\scshape{Tesi di laurea seguite come correlatore}}

\hspace*{-1cm}
\begin{tabular}{lllp{9cm}}

  \icon{0.05}{graduationcap.png} \quad 2018 & Sofia Farina          & L-DM270  & \emph{A physical interpretation of network laplacian: role of perturbations and masses}\\
  \icon{0.05}{graduationcap.png} \quad 2018 & Giovanni Marangi      & L-DM270  & \emph{Teoria dei network applicata alle strutture proteiche}\\
  \icon{0.05}{graduationcap.png} \quad 2018 & Lorenzo Dall'Olio     & L-DM270  & \emph{Funzionamento di un pulsossimetro ed analisi di serie temporali pulsossimetriche}\\
  \icon{0.05}{graduationcap.png} \quad 2018 & Mattia Ceccarelli     & L-DM270  & \emph{Analisi della complessità di reti neurali generate tramite algoritmi genetici}\\
  \icon{0.05}{degree.png}        \quad 2019 & Alex Baroncini        & LM-DM270 & \emph{Sviluppo ed ottimizzazione di algoritmi per super-risoluzione ed object detection mediante deep neural network}\\
  \icon{0.05}{graduationcap.png} \quad 2019 & Davide Ravaglia       & L-DM270  & \emph{Modelling social behavior of Drosophila Melanogaster under the effect of drugs}\\
  \icon{0.05}{degree.png}        \quad 2019 & Daniele Dall'Olio     & LM-DM270 & \emph{Applicazione di un algoritmo d’apprendimento basato su sistemi fuori dall’equilibrio a dati di Genome Wide Association}\\
  \icon{0.05}{degree.png}        \quad 2020 & Alessandro D'Agostino & LM-DM270 & \emph{Dataset generation for the training of Neural Networks oriented toward histological image segmentation}\\
  \icon{0.05}{degree.png}        \quad 2020 & Mattia Ceccarelli     & LM-DM270 & \emph{Optimization and applications of deep learning algorithms for super-resolution in MRI}\\
  \icon{0.05}{graduationcap.png} \quad 2020 & Diego Cardinali       & L-DM270  & \emph{Classification of Clausocalanus Furcatus motion utilizing the random walk theory}\\
  \icon{0.05}{graduationcap.png} \quad 2021 & Davide De Paoli       & L-DM270  & \emph{Reti neurali artificiali e apprendimenti basati sulla biofisica dei neuroni}\\
  \icon{0.05}{degree.png}        \quad 2021 & Riccardo Biondi       & LM-DM270 & \emph{Implementation of an Automated Pipeline for the Identification of Ground Glass Opacities on CT Scans of Patients Affected by COVID-19}\\
  \icon{0.05}{degree.png}        \quad 2021 & Davide Panzeri        & LM-DM270 & \emph{AI based prediction for brightfield and stained anatomopathology}\\
  \icon{0.05}{degree.png}        \quad 2021 & Laura Verzellesi      & LM-DM270 & \emph{Machine Learning methods for hepatocellular malignancies segmentation and MVI prediction}\\
  \icon{0.05}{degree.png}        \quad 2021 & Giuseppe Filitto      & LM-DM270 & \emph{Implementation of an automated pipeline to predict the response to neoadjuvant chemo-radiotherapy of patients affected by colorectal cancer}\\
  \icon{0.05}{degree.png}        \quad 2022 & Caterina Faccioli     & LM-DM270 & \emph{Spatial analysis in pathomics: a network based method applied on fluorescence microscopy}\\
  %\icon{0.05}{graduationcap.png} \quad 2022 & Andrea Corvina        & L-DM270  & \emph{}\\
  %\icon{0.05}{degree.png}        \quad 2022 & Stefano Bianchi       & LM-DM270 & \emph{}\\
  %\icon{0.05}{degree.png}        \quad 2022 & Giancarlo Cuticchia   & LM-DM270 & \emph{}\\
  %\icon{0.05}{degree.png}        \quad 2022 & Daniele Buschi        & LM-DM270 & \emph{}\\
  %\icon{0.05}{degree.png}        \quad 2022 & Chiara Vece           & LM-DM270 & \emph{}\\

\end{tabular}


\vspace*{0.5cm}
\section*{\scshape{Collaborazioni}}

\begin{tabular}{llp{12cm}}

  aa 2014             & Tesi Triennale  & Collaborazione con il team di ricerca MRPM guidato dalla Prof.ssa Paola Fantazzini dell'Università di Bologna, elaborando dati di imaging cellulare da me raccolti mediante microscopia ottica in contrasto di fase. Il mio contributo è stato evidenziato nella sezione Acknowledgments del paper redatto, dal titolo \emph{Water compartmentalization, cell viability and morphology changes monitored under stress by 1H-NMR relaxometry and phase contrast optical microscopy} (L. Brizi et al.), pubblicato su \emph{Journal of Physics D: Applied Physics}. \\
  aa 2016             & Tesi Magistrale & Collaborazione con il Data Center INFN-CNAF per l'implementazione ed ottimizzazione di algoritmi per l'analisi di Big Data genomici su architetture di calcolo distribuito.\\
  2016\textemdash2017 & Dottorato       & Collaborazione con Unipol Assicurazioni al fine di ottimizzare la loro rete peritale in termini di disposizione territoriale ed efficienza dei periti.\\
  2016\textemdash2017 & Dottorato       & Collaborazione con il comune di Venezia e Telecom allo studio dei flussi pedonali all'interno del comune veneziano, mediante dati geo-referenziati di telefonia mobile. \\
  2017\textemdash2018 & Dottorato       & Collaborazione con Canon e Fabbrica Digitale per lo sviluppo di tecnologie atte al rilevamento dei flussi pedonali nel comune di Venezia, che ci ha permesso di vincere il bando per il progetto \emph{Venice} in tematica \emph{smart cities}. \\
  2018\textemdash2019 & Dottorato       & Collaborazione con il centro INFN di Roma \emph{La Sapienza} allo sviluppo di algoritmi per la \emph{sentimental analysis} e il \emph{natural language processing}.\\
  2019                & Assegnista      & Collaborazione con l'unità ospedaliera di Oftalmologia del policlinico IRCCS Sant'Orsola - Malpighi allo sviluppo di \emph{Decision Support System} per l'analisi di immagini acquisite mediante lampada a fessura.\\
  2019\textemdash2022 & Assegnista      & Collaborazione con l'Università degli Studi di Milano - Bicocca allo sviluppo di sistemi automatici per la standardizzazione della raccolta dati in microscopia a fluorescenza e seconda armonica in immagini istopatologiche.\\
  2019\textemdash2022 & Assegnista      & Collaborazione con l'unità ospedaliera di Dermatologia del policlinico IRCCS Sant'Orsola - Malpighi allo sviluppo di \emph{Decision Support System} per la segmentazione e valutazione di immagini istopatologiche.\\
  2020\textemdash2022 & Assegnista      & Collaborazione con l'unità ospedaliera di Dermatologia del policlinico IRCCS Sant'Orsola - Malpighi per la gestione ed analisi di immagini di ferite aperte e piaghe.\\
  2020\textemdash2022 & Assegnista      & Collaborazione con l'unità ospedaliera di Radiologia del policlinico IRCCS Sant'Orsola - Malpighi per la gestione ed analisi di immagini MRI, CT e PET.\\
  2020\textemdash2022 & Assegnista      & Collaborazione con l'unità di ematologia di Humanitas Research Hospital per l'analisi di immagini istologiche.\\
  2021\textemdash2022 & Assegnista      & Collaborazione con l'unità ospedaliera di Microscopia Elettronica del policlinico IRCCS Sant'Orsola - Malpighi per l'analisi di immagini TEM.\\

\end{tabular}


\vspace*{0.5cm}
\section*{\scshape{Pubblicazioni}}

\begin{itemize}

  \itemicon{0.05}{article.png} A. Bazzani, A. Fabbri, C. Mizzi, S. Rambaldi, F. Bertini, S. Sinigardi, \textbf{N. Curti}, R. Luzi, G. Venturi, D. Micheli, G. Muratore and A. Vannelli, \emph{Unraveling pedestrian mobility on a road network using ICTs data during great tourist events}, \emph{EPJ Data Science}, \url{10.1140/epjds/s13688-018-0168-2} (2018 Oct 22)

  \itemicon{0.05}{article.png} \textbf{N. Curti} \& E. Giampieri, A. Ferraro, M. C. Vistoli, E. Ronchieri, D. Cesini, B. Martelli, D. C. Duma and G. Castellani, \emph{Cross-Environment comparison of a bioinformatics pipeline: perspectives for hybrid computations}, \emph{Euro-Par 2018: Parallel Processing Workshops}, \url{10.1007/978-3-030-10549-5_50} (2018 Dec 12)

  \itemicon{0.05}{article.png} V. Boccardi, L. Paolacci, D. Remondini, E. Giampieri, G. Poli, \textbf{N. Curti}, R. Cecchetti, A. Villa, S. Brancorsini, P. Mecocci, \emph{Cognitive decline and Alzheimer's disease in the old age: sex influence on a \quotes{cytokinome signature}}, \emph{Journal of Alzheimer's Disease}, \url{10.3233/JAD-190480} (2019 Nov 26)

  \itemicon{0.05}{article.png} M. Malvisi \& \textbf{N. Curti}, D. Remondini, F. Palazzo, J. L. Williams, G. Pagnacco, G. Minozzi, \emph{Combinatorial Discriminant Analysis applied to RNAseq data reveals a set of 10 transcripts as signatures of infection of cattle with Mycobacterium avium subsp. paratuberculosis}, \emph{Animals}, \url{10.3390/ani10020253} (2020 Feb 5)

  \itemicon{0.05}{article.png} L. Dall'Olio, \textbf{N. Curti}, D. Remondini, Y. Safi Harb, F. W. Asselbergs, G. Castellani, H. Uh, \emph{Prediction of vascular ageing based on smartphone acquired PPG signals}, \emph{Scientific Reports}, \url{10.1038/s41598-020-76816-6} (2020 Nov 12)

  \itemicon{0.05}{article.png} D. Dall'Olio \& \textbf{N. Curti}, E. Fonzi, C. Sala, D. Remondini, G. Castellani, E. Giampieri, \emph{Impact of Concurrency on the Performance of a Whole Exome Sequencing Pipeline}, \emph{BMC Bioinformatics}, \url{10.1186/s12859-020-03780-3} (2021 Feb 9)

  \itemicon{0.05}{article.png} \textbf{N. Curti} \& E. Giampieri, F. Guaraldi, F. Bernabei, L. Cercenelli, G. Castellani, P. Versusa, E. Marcelli, \emph{A fully automated pipeline for a robust conjunctival hyperemia estimation}, \emph{Applied Sciences}, \url{10.3390/app11072978} (2021 Mar 26)

  \itemicon{0.05}{article.png} R. Biondi \& \textbf{N. Curti}, F. Coppola, E. Giampieri, G. Vara, M. Bartoletti, A. Cattabriga, M. A. Cocozza, F. Ciccarese, C. De Benedittis, L. Cercenelli, B. Bortolani, E. Marcelli, L. Pierotti, L. Strigari, P. Viale, R. Golfieri and G. Castellani, \emph{Classification Performance for COVID patient prognosis from automatic AI segmentation – a single center study}, \emph{Applied Science}, \url{10.3390/app11125438} (2021 June 11)

  \itemicon{0.05}{article.png} S. Valente, \textbf{N. Curti}, E. Giampieri, V. Randi, C. Donadei, M. Buzzi, P. Versura, \emph{Impact of blood source and component manufacturing on neurotrophin content and in vitro cell wound healing}, \emph{Blood Transfusion}, \url{10.2450/2021.0116-21} (2021 Aug 3)

  \itemicon{0.05}{article.png} L. Cercenelli, M. Zoli, B. Bortolani, \textbf{N. Curti}, D. Gori, A. Rustici, D. Mazzatenta, E. Marcelli, \emph{3D virtual modeling for morphological characterization of pituitary tumors: preliminary results on its predictive role in tumor resection rate}, \emph{Applied Sciences}, \url{10.3390/app12094275} (2022 Apr 23)

  \itemicon{0.05}{article.png} L. Spagnoli, M. F. Morrone, E. Giampieri, G. Paolani, M. Santoro, \textbf{N. Curti}, F. Coppola, F. Ciccarese, G. Vara, N. Brandi, R. Golfieri, P. Viale, M. Bartoletti, L. Strigari, \emph{Outcome prediction for Sars-CoV-2 patients using machine learning modelling of clinical, radiological and radiomic features derived from chest CT images}, \emph{Applied Sciences}, \url{10.3390/app12094493} (2022 Apr 28)

  \itemicon{0.05}{article.png} G. Filitto, F. Coppola, \textbf{N. Curti$^\dagger$}, E. Giampieri, D. Dall’Olio, A. Merlotti, A. Cattabriga, M.A. Cocozza, M.T. Tomassoni, D. Remondini, L. Pierotti, L. Strigari, D. Cuicchi, A. Guido, K. Rihawi, A. D'Errico, F. Di Fabio, G. Poggioli, A.G. Morganti, L. Ricciardiello, R. Golfieri, G. Castellani, \emph{Automated Prediction of the Response to Neoadjuvant Chemoradiotherapy in Patients Affected by Rectal Cancer}, \emph{Cancers}, \url{10.3390/cancers14092231} (2022 Apr 29)

  \itemicon{0.05}{article.png} L. Squadrani \& \textbf{N. Curti}, E. Giampieri, D. Remondini, B. Blais, G. Castellani, \emph{Effectiveness of biological inspired neural network models in learning and patterns memorization}, \emph{Entropy}, \url{10.3390/e24050682} (2022 May 12)

  \itemicon{0.05}{article.png} G. Carlini \& \textbf{N. Curti}, S. Strolin, S. Fanti, D. Remondini, C. Nanni, L. Strigari, and G. Castellani, \emph{Prediction of overall survival in cervical cancer patients using PET/CT radiomic features}, \emph{Applied Science}, \url{10.3390/app12125946} (2022 June 10)

  \itemicon{0.05}{article.png} E. Dika \& \textbf{N. Curti}, E. Giampieri, G. veronesi, C. Misciali, C. Ricci, G. Castellani, A. Patrizi, E. Marcelli, \emph{Advantages of manual and automatic computer-aided compared to traditional histopathological diagnosis of melanoma: a pilot study}, \emph{Pathology - Research and Practice}, \url{10.1016/j.prp.2022.154014} (2022 July 8)

  \itemicon{0.05}{article.png} \textbf{N. Curti} \& G. Veronesi, E. Dika, C. Misciali, E. Marcelli, E. Giampieri, \emph{Breslow thickness: geometric interpretation, potential pitfalls, and computer automated estimation}, \emph{Pathology - Research and Practice}, \url{10.1016/j.prp.2022.154117} (2022 Sep 5)

 \itemicon{0.05}{article.png} S. Valente, C. Ciavarella, G. Astolfi, E. Bergantin, \textbf{N. Curti}, M. Buzzi, L. Fontana, P. Versura, \emph{Impact of Freeze-Drying of Cord Blood (CB) Serum (S) and Platelet Rich Plasma (CB-PRP) preparations on Growth Factor content and in vitro cell wound healing}, \emph{International Journal of Molecular Sciences}, \url{10.3390/ijms231810701} (2022 Sep 14)

\end{itemize}


% \vspace*{0.5cm}
% \section*{\scshape{Pubblicazioni - Work in progress}}

% \begin{itemize}

%   \itemicon{0.05}{article.png} I. Budimir, E. Giampieri, E. Saccenti, M. S. Diez, M. Tarozzi, D. Dall'Olio, A. Merlotti, \textbf{N. Curti}, D. Remondini, G. Castellani, C. Sala, \emph{Intraspecies characterization of bacteria via evolutionary modeling of protein domains}, \emph{Scientific Reports}, \url{???} (2022 ???)

%   \itemicon{0.05}{article.png} \textbf{N. Curti} \& G. Levi, E. Giampieri, G. Castellani, D. Remondini, \emph{A network approach for low-dimensional signatures from high-throughput data}, \emph{Scientific Reports}, \url{???} (2022 ???)

%   \itemicon{0.05}{article.png} C. Fiscone \& \textbf{N. Curti}, M. Ceccarelli, D. Remondini, C. Testa, R. Lodi, C. Tonon, D. N. Manners, G. Castellani, \emph{Generalizing the Enhanced-Deep-Super-Resolution neural network to brain MR images: a retrospective study on the Cam-CAN dataset}, \emph{???}, \url{???} (2022 ???)

%   \itemicon{0.05}{article.png} \textbf{N. Curti} \&, Y. Merli, C. Zengarini, A. Merlotti, D. Dall'Olio, E. Marcelli, G. Castellani, T. Bianchi, E. Giampieri, \emph{Effectiveness of semi-supervised active learning in automated wound image segmentation}, \emph{Experimental Dermatology}, \url{???} (2022 ???)

%   \itemicon{0.05}{article.png} \textbf{N. Curti} \& Y. Merli, C. Zengarini, A. Merlotti, D. Dall'Olio, E. Marcelli, G. Castellani, T. Bianchi, E. Giampieri, \emph{Automated prediction of Photographic Wound Assessment Tool in Chronic Wound images}, \emph{???}, \url{???} (2022 ???)

%   \itemicon{0.05}{article.png} \textbf{N. Curti} \& G. Carlini, S. Valente, E. Giampieri, A. Merlotti, D. Dall'Olio, C. Sala, E. Marcelli, G. Castellani, G. Pasquinelli, \emph{Fully automated estimation of glomerular basement membrane thickness via active semi-supervised learning model}, \emph{???}, \url{???} (2022 ???)

%   \itemicon{0.05}{article.png} R. Biondi, G. Carlini, \textbf{N. Curti$^\dagger$}, M. Renzulli, C. Sala, F. Coppola, E. Giampieri, G. Vara, A. Cattabriga, M. A. Cocozza, L. V. Pastore, A. Palmeri, M. Cescon, M. Ravaioli, G. Castellani, R. Golfieri, \emph{Automated prediction of MicroVascular Invasion in HepatoCellular Carcinomas Liver Tumor}, \emph{???}, \url{???} (2022 ???)

%   \itemicon{0.05}{article.png} G. Carlini, C. Gaudiano, R. Biondi, R. Golfieri, \textbf{N. Curti$^\dagger$}, D. Caruso, A. Merlotti, D. Dall'Olio, C. Sala, S. Pandolfi, D. Remondini, B. Bortolani, E. Marcelli, F. Coppola, G. Castellani, \emph{Effectiveness of radiomic ZOT features in the automated dis-crimination of oncocytoma from clear cell renal cancer}, \emph{???}, \url{???} (2022 ???)

% \end{itemize}


\vspace*{0.5cm}
\section*{\scshape{Open Access Archive}}

\begin{itemize}

  \itemicon{0.05}{article.png} \textbf{N. Curti}, E. Giampieri, G. Levi, G. Castellani, D. Remondini, \emph{DNetPRO: A network approach for low-dimensional signatures from high-throughput data}, \emph{BioRxiv}, \url{10.1101/773622} (2019 Sep 19)

  \itemicon{0.05}{code.png} \textbf{N. Curti} \& D. Dall'Olio, D. Remondini, G. Castellani, E. Giampieri, \emph{rFBP: Replicated Focusing Belief Propagation algorithm}, \emph{Journal of Open Source Software}, \url{10.21105/joss.02663} (2020 Oct 20)

  \itemicon{0.05}{code.png} R. Biondi \& \textbf{N. Curti}, E. Giampieri, G. Castellani, \emph{COVID-19 Lung Segmentation}, \emph{Journal of Open Source Software}, \url{10.21105/joss.03447} (2021 Sep 30)

\end{itemize}


\vspace*{0.5cm}
\section*{\scshape{Conference abstract}}

\begin{itemize}

  \itemicon{0.05}{abstract.png} D. Dall'Olio, \textbf{N. Curti}, A. Bazzani, D. Remondini, G. Castellani, \emph{Classification of Genome Wide Association data by Belief Propagation Neural network}, \emph{CCS 2019}

  \itemicon{0.05}{abstract.png} C. Mengucci \& \textbf{N. Curti}, E. Giampieri, G. Castellani, D. Remondini, \emph{Introducing the Complex Human Interactions in MEdical Records and Atlases Network - CHIMERA}, \emph{CCS 2019}

  \itemicon{0.05}{poster.png} C. Fiscone, \textbf{N. Curti}, M. Ceccarelli, D. N. Manners, G. Castellani, R. Lodi, D. Remondini, C. Tonon, C. Testa, \emph{Super Resolution of T$_1$w and T$_2$w MRI using deep neural networks: brain images from CamCan dataset}, \emph{GIDRM 2020}, \emph{ISMRM 2021}, [poster session]

  \itemicon{0.05}{abstract.png} C. Fiscone, \textbf{N. Curti}, M. Ceccarelli, D. N. Manners, G. Castellani, R. Lodi, D. Remondini, C. Tonon, C. Testa, \emph{Exploring the use of Enhanced-Deep-Super-Resolution neural network: a retrospective study on the CamCAN brain MRI Dataset}, \emph{ISMRM 2021}

  \itemicon{0.05}{abstract.png} \textbf{N. Curti}, E. Giampieri, D. Dall'Olio, C. Sala, G. Castellani, \emph{Semi-supervised active learning in automated wound image segmentation via smartphone mobile App}, \emph{ICMMB 2022}

\end{itemize}


\vspace*{0.5cm}
\section*{\scshape{Open Source Software}}

\begin{tabular}{cclp{11cm}}

  \href{https://github.com/Nico-Curti/DNetPRO}{\icon{0.025}{github_logo.png}} & \icon{0.025}{cpp.png} \icon{0.025}{python.png}    & \scshape{DNetPRO}         & \emph{Discriminant Analysis with Network PROcessing}                        \\
  \href{https://github.com/Nico-Curti/NumPyNet}{\icon{0.025}{github_logo.png}} & \icon{0.025}{python.png}                         & \scshape{NumPyNet}        & \emph{Neural Networks library in pure numpy}                                \\
  \href{https://github.com/Nico-Curti/Openslide}{\icon{0.025}{github_logo.png}} & \icon{0.025}{cpp.png} \icon{0.025}{python.png}  & \scshape{Openslide}       & \emph{C library for reading virtual slide images}                           \\
  \href{https://github.com/Nico-Curti/Byron}{\icon{0.025}{github_logo.png}} & \icon{0.025}{cpp.png} \icon{0.025}{python.png}      & \scshape{Byron}           & \emph{Build your own neural network}                                        \\
  \href{https://github.com/Nico-Curti/rFBP}{\icon{0.025}{github_logo.png}} & \icon{0.025}{cpp.png} \icon{0.025}{python.png}       & \scshape{rFBP}            & \emph{Replicated Focusing Belief Propagation algorithm}                     \\
  \href{https://github.com/Nico-Curti/plasticity}{\icon{0.025}{github_logo.png}} & \icon{0.025}{cpp.png} \icon{0.025}{python.png} & \scshape{plasticity}      & \emph{Unsupervised Neural Networks with biological-inspired learning rules} \\
  \href{https://github.com/Nico-Curti/scorer}{\icon{0.025}{github_logo.png}} & \icon{0.025}{cpp.png} \icon{0.025}{python.png}     & \scshape{scorer}          & \emph{Machine Learning Scorer Library}                                      \\
  \href{https://github.com/Nico-Curti/ParseArgs}{\icon{0.025}{github_logo.png}} & \icon{0.025}{cpp.png}                           & \scshape{ParseArgs}       & \emph{Simple command line parser in C++}                                    \\
  \href{https://github.com/Nico-Curti/easyDAG}{\icon{0.025}{github_logo.png}} & \icon{0.025}{cpp.png}                             & \scshape{easyDAG}         & \emph{Simple template DAG scheduler in c++}                                 \\
  \href{https://github.com/Nico-Curti/genetic}{\icon{0.025}{github_logo.png}} & \icon{0.025}{cpp.png}                             & \scshape{genetic}         & \emph{Examples about genetic algorithms for parallel computing}             \\
  \href{https://github.com/Nico-Curti/shut}{\icon{0.025}{github_logo.png}} & \icon{0.020}{bash.jpg} \icon{0.05}{pwsh.png}         & \scshape{shut}            & \emph{Shell Utilities and Installers for no root users}                     \\
  \href{https://github.com/Nico-Curti/cardio}{\icon{0.025}{github_logo.png}} & \icon{0.025}{python.png}                           & \scshape{cardio}          & \emph{Pulse oximetry data processing and classification}                    \\
  \href{https://github.com/Nico-Curti/BlendNet}{\icon{0.025}{github_logo.png}} & \icon{0.025}{python.png}                         & \scshape{BlendNet}        & \emph{Network viewer with Blender support}                                  \\
  \href{https://github.com/Nico-Curti/SysDyn}{\icon{0.025}{github_logo.png}} & \icon{0.025}{cpp.png} \icon{0.025}{python.png}     & \scshape{SysDyn}          & \emph{System Dynamics script utilities}                                     \\
  \href{https://github.com/Nico-Curti/nocs}{\icon{0.025}{github_logo.png}} & \icon{0.025}{cpp.png}                                & \scshape{nocs}            & \emph{NOCS (Not Only Colliding Spheres) exact 2D gas dynamics framework}    \\
  \href{https://github.com/Nico-Curti/rSGD}{\icon{0.025}{github_logo.png}} & \icon{0.025}{cpp.png} \icon{0.025}{python.png}       & \scshape{rSGD}            & \emph{Replicated Stochastic Gradient Descent algorithm}                     \\
  \href{https://github.com/Nico-Curti/Walkers}{\icon{0.025}{github_logo.png}} & \icon{0.025}{cpp.png} \icon{0.025}{python.png}    & \scshape{Walkers}         & \emph{Random Walk and Optimizer Simulator}                                  \\
  \href{https://github.com/Nico-Curti/CryptoSocket}{\icon{0.025}{github_logo.png}} & \icon{0.025}{python.png}                     & \scshape{CryptoSocket}    & \emph{TCP/IP Client Server with RSA cryptography}                           \\
  \href{https://github.com/Nico-Curti/FiloBluService}{\icon{0.025}{github_logo.png}} & \icon{0.025}{python.png}                   & \scshape{FiloBluService}  & \emph{FiloBlu Service Manager for text message processing}                  \\
  \href{https://github.com/Nico-Curti/anonymize-slide}{\icon{0.025}{github_logo.png}} & \icon{0.025}{python.png}                  & \scshape{anonymize-slide} & \emph{Delete the label from a whole-slide image}                            \\

\end{tabular}


\vspace*{0.5cm}
\section*{\scshape{Attività Scientifica}}

Durante il corso di laurea magistrale ho avuto modo di approfondire diverse tematiche di ricerca nel campo della biofisica e dei sistemi complessi.
Nel campo delle \emph{complex network} ho lavorato, sotto la supervisione del Prof. Daniel Remondini dell'Università di Bologna, sul tema della ricostruzione della struttura tridimensionale proteica a partire dalle mappe di contatto.
In collaborazione con il prof. Gastone Castellani dell'Università di Bologna, ho approfondito le mie conoscenze sui modelli biologici ed evolutivi, studiando modelli di reazione-diffusione ed integrazione di equazioni differenziali deterministiche e stocastiche.
In relazione anche a questi modelli, in collaborazione con il prof. Armando Bazzani dell'Università di Bologna, ho studiato la teoria sulla ricostruzione degli attrattori del moto, sviluppando algoritmi che permettessero di sfruttare il Teorema di Takens e quindi caratterizzare l'eventuale attrattore a partire da una singola serie temporale di dati.

Durante il lavoro di tesi magistrale mi sono occupato dell'implementazione e benchmarking dell'algoritmo DNetPRO, ideato dal team di ricerca di biofisica dell'Università di Bologna, per l'analisi di Big Data genomici.
Durante il lavoro ho potuto approfondire le mie conoscenze di machine learning, con particolare attenzione alla features extraction e features selection.
Ho applicato questo algoritmo a dati high-throughput comprendenti microRNA, mRNA, Copy Number Variation e Reverse Phase Protein array.
Una versione differente dell'algoritmo l'ho proposta ed utilizzata anche per le elaborazioni di dati di \emph{gene expression} provenienti dalla collaborazione con l'\emph{Institute of Agricultural Biology and Biotechnology} di Lodi: per far fronte a diverse problematiche intrinseche nei dati ho sviluppato una versione alternativa dell'algoritmo, utilizzando metodi di teoria dei network per eseguire un'ingente riduzione della dimensionalità del problema ed estrarre una signature di geni che ha riscotrato valenza anche sul piano biologico.

Nel mio primo anno di dottorato ho potuto approfondire le mie competenze informatiche e algoritmiche.
In questo periodo ho anche avuto modo di applicare le più moderne tecniche di calcolo parallelo e distribuito, eseguendo analisi su dati anche di ingenti dimensione.
Gli algoritmi da me sviluppati sono stati applicati anche per l'analisi di dati forniti da Unipol Assicurazioni, ottenendo un'ottimizzazione dell'intera rete peritale italiana.
In parallelo a ciò ho iniziato a studiare le tecniche di apprendimento su rete neurale, concentrandomi sul modello di \emph{Replicated Focusing Belief Propagation}.
L'ottimizzazione ed implementazione di tale modello mi hanno permesso di analizzare i dati provenienti dall'intera rete stradale regionale emiliana e di sviluppare un metodo per la previsione degli ingorghi stradali automobilistici, in grado di prevedere con un anticipo di 30 minuti il verificarsi dell'ingorgo.
Insieme al flusso automobilistico ho anche iniziato ad affrontare la tematica dei flussi pedonali, riscontrando che le stesse metodologie da me precedentemente applicate per l'analisi dei geni potevano essere utilizzate anche per rilevare le strutture fondamentali del flusso pedonale cittadino.
In particolare, sfruttando la collaborazione con il comune di Venezia, ho potuto ricostruire ed analizzare i flussi pedonali monitorati in giorni di intensa affluenza e fornire previsioni sull'andamento dei loro trend.

Nel mio secondo anno di dottorato ho collaborato con il centro di calcolo CNAF per l'implementazione, ottimizzazione e monitoraggio di pipeline di bioinformatica dimostrando l'utilizzo ed efficienza di architetture low-power per l'analisi di intere catene di genoma umano.
Ho collaborato inoltre con \emph{Canon} e \emph{Fabbrica Digitale} per lo sviluppo di un sistema automatico di telecamere per il monitoring dei flussi pedonali.
Studiando diversi modelli di \emph{Artificial Intelligence} ho utilizzato ed ottimizzato un modello a rete neurale a convoluzione per il riconoscimento dei pedoni in sequenze video ed il loro conteggio e tracking.
In questo lavoro ho avuto modo di affrontare le problematiche relative al controllo remoto di telecamere e riguardanti l'implementazione di codice per GPUs.
Questo lavoro ha anche portato allo sviluppo di una nuova libreria per lo sviluppo di reti neurali interamente ottimizzata per il calcolo multi-threading che ha ottenuto performance di calcolo spesso superiori rispetto alle implementazioni standard delle medesime architetture.

Nel mio terzo anno di dottorato ho collaborato con l'Università Sapienza di Roma all'interno di un progetto di \emph{Natural Language Processing} e \emph{Sentiment Analysis} per la classificazione dei messaggi di chat tra medico e paziente.
All'interno di questo progetto mi sono occupato dell'automatizzazione delle comunicazione tra app di messaggistica e server centrale di storage, sviluppando un servizio/demone in grado di estrarre le informazioni all'interno del database, processarle mediante algoritmi basati su reti neurali e trasmettere lo score all'app di messaggistica.
Questo lavoro mi ha permesso di affinare le mie competenze sulla gestione di grandi quantità di dati e di sviluppare nuovi algoritmi anche per l'analisi del linguaggio naturale.
Tali competenze sono state poi impiegate per lo sviluppo del progetto \emph{CHIMeRA} (\emph{Complex Human Interaction of MEdical Records and Atlases}) nel quale, mediante tecniche di web-scraping, ho lavorato sull'estrazione di dati da interfacce web-html andando a minare e ri-organizzare grandi (e diversi) database pubblici.
Per il progetto, mi sono occupato anche della gestione ed armonizzazione dei database di informazioni biomediche, arrivando a creare una struttura dati NoSQL (o network-of-networks) sul quale poter eseguire query e studiare legami tra le informazioni raccolte.

Nel mio primo anno da assegnista di ricerca presso il policlinico Sant'Orsola - Malpighi mi sono occupato della gestione e organizzazione di dataset medici al fine di poterli stoccare all'interno di server esterni all'area ospedaliera secondo le attuali normative sulla privacy dei dati.
In parallelo, ho collaborato con l'unità ospedaliera di Dermatologia allo sviluppo di un sistema automatico per l'analisi di \emph{Whole Slide Images} (WSI).
Durante questo progetto mi sono potuto specializzare nell'analisi di immagini istologiche, contribuendo alla realizzazione di un software per la gestione, anonimizzazione ed analisi mediante tecniche di computer vision di WSI.
In particolare, ho sviluppato modelli a rete neurale per la segmentazione e classificazione di cellule e tessuti all'interno di WSI, andando ad applicare tecniche di artificial intelligence al campo dell'istopatologia.
Inoltre, ho avuto modo di collaborare con l'unità ospedaliera di Oftalmologia, ed in particolare con la prof.ssa Piera Versura, sull'analisi di immagini acquisite mediante lampada a fessura.
In questo lavoro ho sviluppato un sistema interamente automatizzato per la predizione del livello di arrossamento congiuntivale, in grado di segmentare ed estrapolare la regione di interesse dalle immagini acquisite, identificare la rete vascolare, estrarne un set di features ed infine prevederne lo score medico richiesto.

Nel mio primo anno da assegnista di ricerca presso il policlinico Sant'Orsola - Malpighi mi sono occupato dell'analisi di immagini radiologiche ed applicazione di tecniche di intelligenza artificiale per la segmentazione di aree tumorali e stratificazione dei pazienti attraverso feature radiomiche.
In particolare, ho potuto lavorare su immagini di NMR, PET e CT relative a diversi distretti anatomici, analizzando pazienti affetti da diverse tipologie di tumore.
In questo lavoro, svolto in collaborazione con la prof.ssa Francesca Coppola, ho lavorato allo sviluppo di \emph{Clinical Decision Support System} (CDSS) utilizzati anche durante la pratica clinica ospedaliera.
In parallelo, ho lavorato all'applicazione di tecniche di \emph{Super Risoluzione} mediante modelli a rete neurale, per il miglioramento della risoluzione spaziale in immagini mediche MRI.

Nel mio primo anno da assegnista di ricerca presso il policlinico Sant'Orsola - Malpighi mi sono occupato dell'analisi di immagini dermatologiche relative a piaghe e ferite aperte.
In collaborazione con il dott. Tommaso Bianchi ed il dott. Yuri Merli, mi sono occupato dello sviluppo di un modello di segmentazione per immagini fotografiche e sua integrazione all'interno di hardware mobile.
In questo lavoro ho collaborato allo sviluppo di una mobile app per la segmentazione ed analisi in tempo reale delle lesioni dermatologiche, sviluppando una nuova tecnica di \emph{Active Semi-Supervised Learning} per l'addestramento di modelli a rete neurale, a partire da piccole quantità di dati annotate.
Un modello analogo è stato applicato anche nel settore delle lesioni animali, in collaborazione con il gruppo di Veterinaria dell'Università di Bologna, estendendo le capacità del modello pre-addestrato anche a nuove tipologie di foto.
In parallelo, mi sono occupato della segmentazione ed analisi di immagini di microscopia elettronica.
In collaborazione con il prof. Gianandrea Pasquinelli, ho sviluppato un modello a rete neurale per l'identificazione e monitoraggio degli spessori di Membrane basali glomerulari.


\vspace*{0.5cm}
\section*{\scshape{Competenze Informatiche}}

\begin{tabular}{lp{3cm}cl}

  \icon{0.05}{cpp.png}     & \textbf{C/C++}       & \priority{100}\priority{100}\priority{100}\priority{100}\priority{100} & ottima        \\
  \icon{0.05}{python.png}  & \textbf{Python}      & \priority{100}\priority{100}\priority{100}\priority{100}\priority{100} & ottima        \\
  \icon{0.075}{latex.png}  & \LaTeX               & \priority{100}\priority{100}\priority{100}\priority{100}\priority{50 } & ottima        \\
  \icon{0.04}{bash.jpg}    & \textbf{Bash}        & \priority{100}\priority{100}\priority{100}\priority{100}\priority{0  } & più che buona \\
  \icon{0.1}{pwsh.png}     & \textbf{Powershell}  & \priority{100}\priority{100}\priority{100}\priority{100}\priority{0  } & più che buona \\
  \icon{0.15}{matlab.png}  & \textbf{Matlab}      & \priority{100}\priority{100}\priority{100}\priority{0  }\priority{0  } & buona         \\
  \icon{0.05}{html.png}    & \textbf{HTML/CSS}    & \priority{100}\priority{100}\priority{100}\priority{0  }\priority{0  } & buona         \\
  \icon{0.05}{js.png}      & \textbf{JavaScript}  & \priority{100}\priority{100}\priority{100}\priority{0  }\priority{0  } & buona         \\
  \icon{0.05}{java.png}    & \textbf{Java}        & \priority{100}\priority{100}\priority{0  }\priority{0  }\priority{0  } & buona         \\
  \icon{0.05}{android.png} & \textbf{Android}     & \priority{100}\priority{100}\priority{0  }\priority{0  }\priority{0  } & buona         \\

\end{tabular}

\vspace*{0.5cm}
\noindent Inoltre ho avuto modo di utilizzare i linguaggi di programmazione \emph{Julia}, \emph{Scala}, software di rendering grafico come \emph{Blender} e \emph{MeshMixer}, software di realtà virtuale come \emph{SteamVR}.


\vspace*{0.5cm}
\section*{\scshape{Competenze Linguistiche}}

\begin{tabular}{lp{3cm}cl}

  \icon{.05}{ita.png} & \textbf{Italiano:} & \priority{100}\priority{100}\priority{100}\priority{100}\priority{100} & Madrelingua \\
  \icon{.05}{eng.png} & \textbf{Inglese:}  & \priority{100}\priority{100}\priority{100}\priority{100}\priority{0}   & livello B2  \\

\end{tabular}


\vspace*{0.5cm}
\quad


\begin{flushright}
Bologna, \today

In fede

\vspace*{0.5cm}

\begin{figure}[hb!]
  \begin{flushright}
    \SignatureImage[scale=0.5]{img/Firma.png}
  \end{flushright}
\end{figure}

\end{flushright}

\vspace*{\fill}
\textbf{Autorizzo al trattamento dei dati personali contenuti in questo documento ai sensi dell'articolo 13 del D. Lgs. 196/2003.}

\end{document}
