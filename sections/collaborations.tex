\IfLanguageName{italian}{
  \section*{\scshape{Collaborazioni}}

  \newcommand\TesiTriennale{Tesi Triennale}
  \newcommand\TesiMagistrale{Tesi Magistrale}
  \newcommand\Dottorato{Dottorato}
  \newcommand\Assegnista{Assegnista}

  \hspace*{-1cm}
  \begin{tabular}{llp{12cm}}

    aa 2014             & \TesiTriennale   & Collaborazione con il team di ricerca MRPM guidato dalla Prof.ssa Paola Fantazzini dell'Università di Bologna, elaborando dati di imaging cellulare da me raccolti mediante microscopia ottica in contrasto di fase. Il mio contributo è stato evidenziato nella sezione Acknowledgments del paper redatto, dal titolo \emph{Water compartmentalization, cell viability and morphology changes monitored under stress by 1H-NMR relaxometry and phase contrast optical microscopy} (L. Brizi et al.), pubblicato su \emph{Journal of Physics D: Applied Physics}. \\
    aa 2016             & \TesiMagistrale  & Collaborazione con il Data Center INFN-CNAF per l'implementazione ed ottimizzazione di algoritmi per l'analisi di Big Data genomici su architetture di calcolo distribuito.\\
    2016\textemdash2017 & Dottorato        & Collaborazione con Unipol Assicurazioni al fine di ottimizzare la loro rete peritale in termini di disposizione territoriale ed efficienza dei periti.\\
    2016\textemdash2017 & \Dottorato       & Collaborazione con il comune di Venezia e Telecom allo studio dei flussi pedonali all'interno del comune veneziano, mediante dati geo-referenziati di telefonia mobile. \\
    2017\textemdash2018 & \Dottorato       & Collaborazione con Canon e Fabbrica Digitale per lo sviluppo di tecnologie atte al rilevamento dei flussi pedonali nel comune di Venezia, che ci ha permesso di vincere il bando per il progetto \emph{Venice} in tematica \emph{smart cities}. \\
    2018\textemdash2019 & \Dottorato       & Collaborazione con il centro INFN di Roma \emph{La Sapienza} allo sviluppo di algoritmi per la \emph{sentimental analysis} e il \emph{natural language processing}.\\
    2019\textemdash2023 & \Assegnista      & Collaborazione con l'unità ospedaliera di Oftalmologia del policlinico IRCCS Sant'Orsola - Malpighi allo sviluppo di \emph{Decision Support System} per l'analisi di immagini acquisite mediante lampada a fessura.\\
    2019\textemdash2022 & \Assegnista      & Collaborazione con l'Università degli Studi di Milano - Bicocca allo sviluppo di sistemi automatici per la standardizzazione della raccolta dati in microscopia a fluorescenza e seconda armonica in immagini istopatologiche.\\
    2019\textemdash2022 & \Assegnista      & Collaborazione con l'unità ospedaliera di Dermatologia del policlinico IRCCS Sant'Orsola - Malpighi allo sviluppo di \emph{Decision Support System} per la segmentazione e valutazione di immagini istopatologiche.\\
    2020\textemdash2023 & \Assegnista      & Collaborazione con l'unità ospedaliera di Dermatologia del policlinico IRCCS Sant'Orsola - Malpighi per la gestione ed analisi di immagini di ferite aperte e piaghe.\\
    2020\textemdash2023 & \Assegnista      & Collaborazione con l'unità ospedaliera di Radiologia del policlinico IRCCS Sant'Orsola - Malpighi per la gestione ed analisi di immagini MRI, CT e PET.\\
    2020\textemdash2023 & \Assegnista      & Collaborazione con l'unità di ematologia di Humanitas Research Hospital per l'analisi di immagini istologiche.\\
    2021\textemdash2023 & \Assegnista      & Collaborazione con l'unità ospedaliera di Microscopia Elettronica del policlinico IRCCS Sant'Orsola - Malpighi per l'analisi di immagini TEM.\\

  \end{tabular}

}{
  \section*{\scshape{Collaborations}}

  \newcommand\TesiTriennale{Bachelor Degree}
  \newcommand\TesiMagistrale{Master Degree}
  \newcommand\Dottorato{Doctorate}
  \newcommand\Assegnista{PhD}

  \hspace*{-1cm}
  \begin{tabular}{llp{12cm}}

    aa 2014             & \TesiTriennale   & Collaboration with the team MRPM leaded by Prof. Paola Fantazzini of the University of Bologna, performing the analysis of contrast-phase microscopy images. My contribution was highlighted in the scientific paper \emph{Water compartmentalization, cell viability and morphology changes monitored under stress by 1H-NMR relaxometry and phase contrast optical microscopy} (L. Brizi et al.), published on \emph{Journal of Physics D: Applied Physics}. \\
    aa 2016             & \TesiMagistrale  & Collaboration with the Data Center INFN-CNAF for the implementation and optimization of algorithms applied to genomic Big Data on distributed computing architectures.\\
    2016\textemdash2017 & Dottorato        & Collaboration with Unipol Assicurazioni for the optimization of the surveyors network in terms of spatial distribution and efficiency of the interventions.\\
    2016\textemdash2017 & \Dottorato       & Collaboration with the city of Venice and Telecom for the analysis of pedestrian mobility on a road network using ICTs data. \\
    2017\textemdash2018 & \Dottorato       & Collaboration with Canon and Fabbrica Digitale for the development of new technologies for the monitoring of pedestrian flows in the framework of \emph{smart cities}. \\
    2018\textemdash2019 & \Dottorato       & Collaboration with INFN Center of Rome \emph{La Sapienza} for the development of algorithms applied to \emph{sentimental analysis} and \emph{natural language processing}.\\
    2019\textemdash2023 & \Assegnista      & Collaboration with the Ophtalmic Unit of the Azienda Ospedaliera IRCSS Sant'Orsola - Malpighi of Bologna for the development of \emph{Decision Support Systems} applied to slit-lamp images.\\
    2019\textemdash2022 & \Assegnista      & Collaboration with the University of Milan - Bicocca for the development of automated systems for the standardization of microscopy and fluorescence histological images.\\
    2019\textemdash2022 & \Assegnista      & Collaboration with the Dermatological Unit of the Azienda Ospedaliera IRCCS Sant'Orsola-Malpighi of Bologna for the development of \emph{Decision Support Systems} for the segmentation and evaluation of histopathological images.\\
    2020\textemdash2022 & \Assegnista      & Collaboration with the Dermatological Unit of the Azienda Ospedaliera IRCCS Sant'Orsola-Malpighi of Bologna for the management and analysis of chronic wound images.\\
    2020\textemdash2023 & \Assegnista      & Collaboration with the Radiological Unit of the Azienda Ospedaliera IRCSS Sant'Orsola-Malpighi of Bologna for the management and analysis of MRI, CT, and PET images.\\
    2020\textemdash2023 & \Assegnista      & Collaboration with the Hematological Unit of the Humanitas Research Hospital for the analysis of histpathological images.\\
    2021\textemdash2023 & \Assegnista      & Collaboration with the Electronic Microscopy Unit of the Azienda Ospedaliera IRCCS Sant'Orsola-Malpighi of Bologna for the analysis of TEM images.\\

  \end{tabular}
}
