\IfLanguageName{italian}{

  Nico Curti nasce a Cattolica (RN, Italia) il 28/09/1992.
  Dopo aver conseguito la laurea triennale in Fisica nel 2014 presso L'Università di Bologna (BO, Italia), consegue la laurea specialistica in \emph{Fisica Applicata} (a beni culturali, ambientali, biologia e medicina) con il voto di 110/110 e lode, presentando il lavoro di tesi dal titolo \emph{Implementazione e benchmarking dell'algoritmo QDANet PRO per l'analisi di Big Data genomici}, svolto sotto la supervisione del Prof. Daniel Remondini e il Prof. Gastone Castellani dell'Università di Bologna.
  Durante il lavoro di tesi, il Dr. Curti ha avuto modo di approfondire le proprie competenze informatiche, focalizzando il suo lavoro sull'implementazione ed ottimizzazione di algoritmi per applicazioni su \emph{high-performance-computers}.

  Successivamente, nel 2019, consegue il dottorato di ricerca in Fisica, difendendo la tesi dal titolo \quotes{Implementation and optimization of algorithms in Biological Big Data Analytics}, sotto la supervisione del Prof. Daniel Remondini, Prof. Gastone Castellani e Prof. Armando Bazzani.
  Durante il lavoro di dottorato il Dr. Curti ha approfondito le sue competenze nel campo dell'\emph{intelligenza artificiale} e \emph{machine learning}, studiando i moderni modelli a rete neurale \emph{deep learning} e le più efficienti tecniche algoritmiche per l'analisi dati.
  In particolare, il Dr. Curti si è occupato dell'analisi di Big Data genomici mediante nuove tecniche di \emph{machine learning} e dell'applicazione di modelli a super risoluzione, object detection e segmentazione su immagini di natura bio-medicale.
  Durante il suo dottorato il Dr. Curti ha avuto l'occasione di collaborare attivamente con numerosi gruppi di ricerca di altri atenei ed è intervenuto nell'elaborazione dati di numerosi progetti con enti privati.

  Nel periodo tra il 2019-2021, Dr. Curti è stato assegnista di ricerca presso il Dipartimento di Medicina Specialistica, Diagnostica e Sperimentale dell'Università di Bologna, sotto la supervisione del Prof. Enrico Giampieri, della Prof.ssa Emanuela Marcelli e del Prof. Gastone Castellani.
  Durante questo periodo ha avuto modo di collaborare attivamente in numerosi progetti di carattere medico, affiancato da diversi membri delle unità ospedaliere.
  In particolare, il Dr. Curti ha lavorato all'analisi di immagini istopatologiche (WSI) per la segmentazione e caratterizzazione di tessuti e cellule, sviluppando \emph{Decision Support System}s in grado di affiancare la metodologia di valutazione tradizione durante la pratica clinica.
  Inoltre ha collaborato con l'unità ospedaliera di Oftalmologia allo sviluppo di una metodologia completamente automatizzata per la valutazione di immagini acquisite mediante lampada a fessura.
  Il Dr. Curti ha svolto numerose ricerche ed applicazioni nel campo dell'\emph{intelligenza artificiale} applicate a immagini radiologiche, comprensive di immagini CT, PET e MRI, realizzando modelli per la segmentazione automatica delle aree tumorali e successiva caratterizzazione mediante elaborazione di features radiomiche.

  Dal 2023 ad oggi, il Dr. Curti risulta ricercatore a tempo determinato tipo a) (junior) presso il Dipartimento di Fisica e Astronomia dell'Università di Bologna, sotto la supervisione del Prof. Daniel Remondini.

} {

  Nico Curti was born in Cattolica (RN, Italy) on 28/09/1992.
  He obtained the bachelor degree in Physics in 2014 at the University of Bologna (BO, Italy) and the Master Degree in \emph{Applied Physics} with the vote of 110/110 with laude, defending the work entitled \emph{Implementazione e benchmarking dell'algoritmo QDANet PRO per l'analisi di Big Data genomici}, developed under the supervision of Prof. Daniel Remondini and Prof. Gastone Castellani.
  During his work of thesis, Dr. Nico Curti deepened his informatic skills, focusing his work on the implementation and optimization of algorithms applied on \emph{high-performance-computers}.

  Later, in 2019, he obtained the PhD in Physics, defending the thesis entitled \quotes{Implementation and optimization of algorithms in Biological Big Data Analytics}, developed under the supervision of Prof. Daniel Remondini, Prof. Gastone Castellani, and Prof. Armando Bazzani at the University of Bologna.
  Dr. Curti collaborated with numerous international research groups during his PhD, and he developed applications of data analysis for several academic and private projects.

  During the years 2019-2021, Dr. Curti was research fellow at the Dept. of Experimental, Diagnostic and Specialty Medicine of Bologna University, under the supervision of Prof. Enrico Giampieri, Prof. Emanuela Marcelli, and Prof. Gastone Castellani.
  Along this period he collaborated to numerous projects about biomedical applications, working side-by-side by expert clinicians.
  In particular, he worked on the analysis of histopathological images (WSI) developing machine learning models for the segmentation and characterization of biological tissues and microscopy analyses.
  The resulting \emph{Decision Support System}s aimed to facilitate and improve the clinical efficiency of the Azienda Ospedaliera IRCCS Sant'Orsola-Malpighi of Bologna.
  Furthermore, he collaborated with the Ophatmological Unit of the Azienda Ospedaliera IRCCS Sant'Orsola-Malpighi of Bologna, developing fully automated solutions for the evaluation of slit lamp images.
  Dr. Curti worked on several scientific research projects involving the application of \emph{artificial intelligence} models on radiological images, involving CT, PET, and MRI formats.
  The resulting models were used for the automated segmentation of cancer volumes and radiomic analysis of the tissues.

  From 2023 to date, Dr. Curti is a junior assistant professor (fixed term) at the Dept. of Physics and Astronomy of Bologna University, under the supervision of Prof. Daniel Remondini.
}
