
\IfLanguageName{italian} {
\section*{\scshape{Attività Scientifica}}

Durante il corso di laurea magistrale ho avuto modo di approfondire diverse tematiche di ricerca nel campo della biofisica e dei sistemi complessi.
Nel campo delle \emph{complex network} ho lavorato, sotto la supervisione del Prof. Daniel Remondini dell'Università di Bologna, sul tema della ricostruzione della struttura tridimensionale proteica a partire dalle mappe di contatto.
In collaborazione con il prof. Gastone Castellani dell'Università di Bologna, ho approfondito le mie conoscenze sui modelli biologici ed evolutivi, studiando modelli di reazione-diffusione ed integrazione di equazioni differenziali deterministiche e stocastiche.
In relazione anche a questi modelli, in collaborazione con il prof. Armando Bazzani dell'Università di Bologna, ho studiato la teoria sulla ricostruzione degli attrattori del moto, sviluppando algoritmi che permettessero di sfruttare il Teorema di Takens e quindi caratterizzare l'eventuale attrattore a partire da una singola serie temporale di dati.

Durante il lavoro di tesi magistrale mi sono occupato dell'implementazione e benchmarking dell'algoritmo DNetPRO, ideato dal team di ricerca di biofisica dell'Università di Bologna, per l'analisi di Big Data genomici.
Durante il lavoro ho potuto approfondire le mie conoscenze di machine learning, con particolare attenzione alla features extraction e features selection.
Ho applicato questo algoritmo a dati high-throughput comprendenti microRNA, mRNA, Copy Number Variation e Reverse Phase Protein array.
Una versione differente dell'algoritmo l'ho proposta ed utilizzata anche per le elaborazioni di dati di \emph{gene expression} provenienti dalla collaborazione con l'\emph{Institute of Agricultural Biology and Biotechnology} di Lodi: per far fronte a diverse problematiche intrinseche nei dati ho sviluppato una versione alternativa dell'algoritmo, utilizzando metodi di teoria dei network per eseguire un'ingente riduzione della dimensionalità del problema ed estrarre una signature di geni che ha riscotrato valenza anche sul piano biologico.

Nel mio primo anno di dottorato ho potuto approfondire le mie competenze informatiche e algoritmiche.
In questo periodo ho anche avuto modo di applicare le più moderne tecniche di calcolo parallelo e distribuito, eseguendo analisi su dati anche di ingenti dimensione.
Gli algoritmi da me sviluppati sono stati applicati anche per l'analisi di dati forniti da Unipol Assicurazioni, ottenendo un'ottimizzazione dell'intera rete peritale italiana.
In parallelo a ciò ho iniziato a studiare le tecniche di apprendimento su rete neurale, concentrandomi sul modello di \emph{Replicated Focusing Belief Propagation}.
L'ottimizzazione ed implementazione di tale modello mi hanno permesso di analizzare i dati provenienti dall'intera rete stradale regionale emiliana e di sviluppare un metodo per la previsione degli ingorghi stradali automobilistici, in grado di prevedere con un anticipo di 30 minuti il verificarsi dell'ingorgo.
Insieme al flusso automobilistico ho anche iniziato ad affrontare la tematica dei flussi pedonali, riscontrando che le stesse metodologie da me precedentemente applicate per l'analisi dei geni potevano essere utilizzate anche per rilevare le strutture fondamentali del flusso pedonale cittadino.
In particolare, sfruttando la collaborazione con il comune di Venezia, ho potuto ricostruire ed analizzare i flussi pedonali monitorati in giorni di intensa affluenza e fornire previsioni sull'andamento dei loro trend.

Nel mio secondo anno di dottorato ho collaborato con il centro di calcolo CNAF per l'implementazione, ottimizzazione e monitoraggio di pipeline di bioinformatica dimostrando l'utilizzo ed efficienza di architetture low-power per l'analisi di intere catene di genoma umano.
Ho collaborato inoltre con \emph{Canon} e \emph{Fabbrica Digitale} per lo sviluppo di un sistema automatico di telecamere per il monitoring dei flussi pedonali.
Studiando diversi modelli di \emph{Artificial Intelligence} ho utilizzato ed ottimizzato un modello a rete neurale a convoluzione per il riconoscimento dei pedoni in sequenze video ed il loro conteggio e tracking.
In questo lavoro ho avuto modo di affrontare le problematiche relative al controllo remoto di telecamere e riguardanti l'implementazione di codice per GPUs.
Questo lavoro ha anche portato allo sviluppo di una nuova libreria per lo sviluppo di reti neurali interamente ottimizzata per il calcolo multi-threading che ha ottenuto performance di calcolo spesso superiori rispetto alle implementazioni standard delle medesime architetture.

Nel mio terzo anno di dottorato ho collaborato con l'Università Sapienza di Roma all'interno di un progetto di \emph{Natural Language Processing} e \emph{Sentiment Analysis} per la classificazione dei messaggi di chat tra medico e paziente.
All'interno di questo progetto mi sono occupato dell'automatizzazione delle comunicazione tra app di messaggistica e server centrale di storage, sviluppando un servizio/demone in grado di estrarre le informazioni all'interno del database, processarle mediante algoritmi basati su reti neurali e trasmettere lo score all'app di messaggistica.
Questo lavoro mi ha permesso di affinare le mie competenze sulla gestione di grandi quantità di dati e di sviluppare nuovi algoritmi anche per l'analisi del linguaggio naturale.
Tali competenze sono state poi impiegate per lo sviluppo del progetto \emph{CHIMeRA} (\emph{Complex Human Interaction of MEdical Records and Atlases}) nel quale, mediante tecniche di web-scraping, ho lavorato sull'estrazione di dati da interfacce web-html andando a minare e ri-organizzare grandi (e diversi) database pubblici.
Per il progetto, mi sono occupato anche della gestione ed armonizzazione dei database di informazioni biomediche, arrivando a creare una struttura dati NoSQL (o network-of-networks) sul quale poter eseguire query e studiare legami tra le informazioni raccolte.

Nel mio primo anno da assegnista di ricerca presso il policlinico Sant'Orsola - Malpighi mi sono occupato della gestione e organizzazione di dataset medici al fine di poterli stoccare all'interno di server esterni all'area ospedaliera secondo le attuali normative sulla privacy dei dati.
In parallelo, ho collaborato con l'unità ospedaliera di Dermatologia allo sviluppo di un sistema automatico per l'analisi di \emph{Whole Slide Images} (WSI).
Durante questo progetto mi sono potuto specializzare nell'analisi di immagini istologiche, contribuendo alla realizzazione di un software per la gestione, anonimizzazione ed analisi mediante tecniche di computer vision di WSI.
In particolare, ho sviluppato modelli a rete neurale per la segmentazione e classificazione di cellule e tessuti all'interno di WSI, andando ad applicare tecniche di artificial intelligence al campo dell'istopatologia.
Inoltre, ho avuto modo di collaborare con l'unità ospedaliera di Oftalmologia, ed in particolare con la prof.ssa Piera Versura, sull'analisi di immagini acquisite mediante lampada a fessura.
In questo lavoro ho sviluppato un sistema interamente automatizzato per la predizione del livello di arrossamento congiuntivale, in grado di segmentare ed estrapolare la regione di interesse dalle immagini acquisite, identificare la rete vascolare, estrarne un set di features ed infine prevederne lo score medico richiesto.

Nel mio primo anno da assegnista di ricerca presso il policlinico Sant'Orsola - Malpighi mi sono occupato dell'analisi di immagini radiologiche ed applicazione di tecniche di intelligenza artificiale per la segmentazione di aree tumorali e stratificazione dei pazienti attraverso feature radiomiche.
In particolare, ho potuto lavorare su immagini di NMR, PET e CT relative a diversi distretti anatomici, analizzando pazienti affetti da diverse tipologie di tumore.
In questo lavoro, svolto in collaborazione con la prof.ssa Francesca Coppola, ho lavorato allo sviluppo di \emph{Clinical Decision Support System} (CDSS) utilizzati anche durante la pratica clinica ospedaliera.
In parallelo, ho lavorato all'applicazione di tecniche di \emph{Super Risoluzione} mediante modelli a rete neurale, per il miglioramento della risoluzione spaziale in immagini mediche MRI.

Nel mio primo anno da assegnista di ricerca presso il policlinico Sant'Orsola - Malpighi mi sono occupato dell'analisi di immagini dermatologiche relative a piaghe e ferite aperte.
In collaborazione con il dott. Tommaso Bianchi ed il dott. Yuri Merli, mi sono occupato dello sviluppo di un modello di segmentazione per immagini fotografiche e sua integrazione all'interno di hardware mobile.
In questo lavoro ho collaborato allo sviluppo di una mobile app per la segmentazione ed analisi in tempo reale delle lesioni dermatologiche, sviluppando una nuova tecnica di \emph{Active Semi-Supervised Learning} per l'addestramento di modelli a rete neurale, a partire da piccole quantità di dati annotate.
Un modello analogo è stato applicato anche nel settore delle lesioni animali, in collaborazione con il gruppo di Veterinaria dell'Università di Bologna, estendendo le capacità del modello pre-addestrato anche a nuove tipologie di foto.
In parallelo, mi sono occupato della segmentazione ed analisi di immagini di microscopia elettronica.
In collaborazione con il prof. Gianandrea Pasquinelli, ho sviluppato un modello a rete neurale per l'identificazione e monitoraggio degli spessori di Membrane basali glomerulari.

\vspace*{0.5cm}
} {

}
